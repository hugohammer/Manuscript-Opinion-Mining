%
% File acl2014.tex
%
% Contact: koller@ling.uni-potsdam.de, yusuke@nii.ac.jp
%%
%% Based on the style files for ACL-2013, which were, in turn,
%% Based on the style files for ACL-2012, which were, in turn,
%% based on the style files for ACL-2011, which were, in turn, 
%% based on the style files for ACL-2010, which were, in turn, 
%% based on the style files for ACL-IJCNLP-2009, which were, in turn,
%% based on the style files for EACL-2009 and IJCNLP-2008...

%% Based on the style files for EACL 2006 by 
%%e.agirre@ehu.es or Sergi.Balari@uab.es
%% and that of ACL 08 by Joakim Nivre and Noah Smith

\documentclass[11pt]{article}
\usepackage{acl2014}
\usepackage{times}
\usepackage{url}
\usepackage{latexsym}

%\setlength\titlebox{5cm}

% You can expand the titlebox if you need extra space
% to show all the authors. Please do not make the titlebox
% smaller than 5cm (the original size); we will check this
% in the camera-ready version and ask you to change it back.


\begin{document}
The parser used in this study is trained on the Norwegian Dependency Treebank (NDT). The NDT is a corpus built up at the National Library of Norway in the period 2011-2013, manually annotated with part-of-speech tags, morphological features, syntactic functions and dependency graphs \cite{Sol14, Sol13}. It consists of approximately 600 000 tokens, equally distributed between Norwegian Bokmål and Nynorsk, the two Norwegian written standards. Only the Bokmål subcorpus has been used here. A large proportion of the NDT is newspaper text, there are also parliament transcripts, government reports and texts with a more colloquial style from blogs. Detailed annotation guidelines in English will be made available in April 2014 \cite{Kinn2013}.




\bibliography{bibl}

\end{document}
