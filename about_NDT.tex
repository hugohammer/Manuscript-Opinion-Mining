\documentclass[a4paper, 12pt]{article}

\usepackage{natbib}
\usepackage{url}
\usepackage{graphicx}
\usepackage{fixltx2e,amsmath}
\usepackage{multirow}
\usepackage{natbib}

\def\bibfontsize{\small}
\def\authorfmt#1{\textsc{#1}}
\def\dcuand{\&}
\let\citeN=\citeasnoun

\MakeRobust{\eqref}

\bibliographystyle{apa}

\linespread{1.0}

\begin{document}
The parser used in this study is trained on the Norwegian Dependency Treebank (NDT). The NDT is a corpus built up at the National Library of Norway in the period 2011-2013, manually annotated with part-of-speech tags, morphological features, syntactic functions and dependency graphs \citep{Sol14, Sol13}. It consists of approximately 600 000 tokens, equally distributed between Norwegian Bokmål and Nynorsk, the two Norwegian written standards. Only the Bokmål subcorpus has been used here. A large proportion of the NDT is newspaper text, there are also parliament transcripts, government reports and texts with a more colloquial style from blogs.




\bibliography{bibl}

\end{document}
